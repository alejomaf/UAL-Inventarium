\section{Balanceo de cargas}
Esta sección creo que merecía estar en la parte final del complemento ya que es una mirada hacia un futuro. Por lo que después de él no hay nada.
\\La escalabilidad de las aplicaciones web ha conseguido mucha representatividad en los últimos tiempos. Podemos ver sitios web gigantescos y acceder a ellos desde todas las partes del mundo.
\\Tenemos ejemplos como Google, Amazon, Facebook. Empresas que sus sitios webs lo son todo y que sin ellos no hubieran cogido la relevancia que han tenido durante todo este tiempo.
\\La escalabilidad la considero un concepto de relevancia debido a que los sistemas tienden a crecer y a ampliarse, siendo optimistas claros. No sabemos si en cuarenta años los docentes y estudiantes se habrán quintuplicado en el Departamento de Informática y en sintonía el uso de la aplicación.
\\Si todo este argumento no convence creo que es mejor recurrir al viejo y clásico: ``Mejor prevenir que curar''.

\subsection{Docker Swarm}
Docker Swarm es una funcionalidad que nos brinda Docker que nos permite gestionar los recursos de nuestros contenedores agrupándolos en un cluster.
\\Esta característica nos ayuda a poder distribuir la carga de trabajo de una aplicación en la red de forma rápida y ahorrando tiempo y recursos de nuestros contenedores. Toda esta gestión se realizaría de forma centralizada.
\\La arquitectura de Swarm es maestro-esclavo. Es decir, cada clúster está formado al menos por un nodo maestro y tantos nodos esclavos como queramos. Mientras que el maestro se encarga de gestionar el clúster y delegar tareas el esclavo se encarga de ejecutar las unidades de trabajo. Como ejemplo pongamos la COVID-19.
\\La COVID supuso un aumento de las infraestucturas red en tiempo récord. Tanto supuso este incremento del uso de internet que había varios momentos al día que este dejaba de funcionar durante un tiempo.
\\Ahora pasemos el ejemplo de infraestucturas red de una página web como la del Aula Virtual de la Universidad de Almería. Esta página implementó un servicio externo para poder realizar videoconferencias que utilizaban bastantes universidades españolas.
\\En un principio todo iba perfecto, ya que se preparó el sistema para una carga grande pero entonces llegó la época de exámenes y la historia se cuenta sola.
\\Una herramienta magnífica que nos ayuda a soportar cargas altas o simplemente distribuir recursos para que unas máquinas no se vean sobrecargadas es Docker Swarm.

\subsection{¿Cómo podemos implementar Docker Swarm?}
Primero de todo necesitamos poder crear otra máquina virtual en nuestra cuenta de Google Cloud. La primera máquina la ubicamos en Estados Unidos pero esta vez la ubicaremos en Finlandia.
\\Procederemos a configurar y a instalarle Docker.
\\TERMINAR CUANDO TENGAS UNA APROXIMACIÓN DEL SITIO WEB QUE VAS A LEVANTAR