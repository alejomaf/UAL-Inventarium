\section{Balanceo de cargas}
Esta sección merecía estar en la parte final del complemento ya que es una mirada hacia un futuro.
\\La escalabilidad de las aplicaciones web ha conseguido mucha representatividad en los últimos tiempos. Se pueden ver sitios web gigantescos y acceder a ellos desde todas las partes del mundo.
\\Teniendo ejemplos como Google, Amazon, Facebook. Empresas que sus sitios webs lo son todo y que sin ellos no hubieran cogido la relevancia que han tenido durante todo este tiempo.
\\La escalabilidad se considera un concepto de relevancia debido a que los sistemas tienden a crecer y a ampliarse, siendo optimistas claros. No se puede saber si en cuarenta años los docentes y estudiantes se habrán quintuplicado en el Departamento de Informática y en sintonía el uso de la aplicación.
\\Si todo este argumento no convence suele ser mejor recurrir al viejo y clásico: ``Mejor prevenir que curar''.

\subsection{Docker Swarm}
Docker Swarm \cite{docker-swarm-book} es una funcionalidad que brinda Docker que permite gestionar los recursos de los contenedores agrupándolos en un cluster.
\\Esta característica ayuda a poder distribuir la carga de trabajo de una aplicación en la red de forma rápida y ahorrando tiempo y recursos de los contenedores. Toda esta gestión se realizaría de forma centralizada.
\\La arquitectura de Swarm es maestro-esclavo. Es decir, cada clúster está formado al menos por un nodo maestro y tantos nodos esclavos como se deseen. Mientras que el maestro se encarga de gestionar el clúster y delegar tareas, el esclavo se encarga de ejecutar las unidades de trabajo. Como ejemplo hablaremos de la COVID-19.
\\La COVID supuso un aumento de las infraestucturas red en tiempo récord. Tanto supuso este incremento del uso de internet que había varios momentos al día que este dejaba de funcionar durante un tiempo.
\\Ahora pasemos el ejemplo de infraestucturas red de una página web como la del Aula Virtual de la Universidad de Almería. Esta página implementó un servicio externo para poder realizar videoconferencias que utilizaban bastantes universidades españolas.
\\En un principio todo iba perfecto, ya que se preparó el sistema para una carga grande pero entonces llegó la época de exámenes y la historia se cuenta sola.
\\Una herramienta magnífica que nos ayuda a soportar cargas altas o simplemente distribuir recursos para que unas máquinas no se vean sobrecargadas es Docker Swarm.

\subsection{¿Cómo puede implementarse Docker Swarm?}
Primero de todo hay que crear otra máquina virtual en la cuenta de Google Cloud. La primera máquina se ubicó en Estados Unidos pero esta vez se ubicará en Finlandia. La configuración se puede ver en la figura \ref{fig:creando-segunda-maquina}.
\begin{figure}
    \centering
    \includegraphics[scale=0.6, keepaspectratio]{imagenes/complemento/docker-swarm/creating_second_virtual_machine.png}
    \caption{Creando segunda máquina virtual}\label{fig:creando-segunda-maquina}
\end{figure}
\\Las dos máquinas virtuales que se tendrían en el entorno Cloud serían las que se pueden ver en la figura \ref{fig:entorno-maquinas-cloud}.
\begin{figure}
    \centering
    \includegraphics[scale=0.45, keepaspectratio]{imagenes/complemento/docker-swarm/virtual_machines_on_google_cloud.png}
    \caption{Entorno de máquinas en Google Cloud}\label{fig:entorno-maquinas-cloud}
\end{figure}
\\Se procederá a configurar y a instalarle Docker.
\\Cuando ya esté instalado hay que dirigirse a la máquina principal que se quiere que actúe como líder, se ejecutará el siguiente comando que se puede ver en la figura \ref{fig:docker-swarm-init}.
\begin{verbatim}
    sudo docker swarm init
\end{verbatim}
\begin{figure}
    \centering
    \includegraphics[scale=0.5, keepaspectratio]{imagenes/complemento/docker-swarm/docker-swarm-init.png}
    \caption{Inicializando el entorno de docker swarm}\label{fig:docker-swarm-init}
\end{figure}
Esto generará un token para poder añadir un nodo al clúster de contenedores. Para poder ingresar a él se escribirá el comando que indica Docker tal como se muestra en la figura \ref{fig:docker-compose-join}.
\begin{figure}
    \centering
    \includegraphics[scale=0.5, keepaspectratio]{imagenes/complemento/docker-swarm/docker-compose-join.png}
    \caption{Añadiendo un nodo esclavo al entorno}\label{fig:docker-compose-join}
\end{figure}
Con esto ya estaría el swarm creado. Pudiendo añadir tantas máquinas como se quieran a ella.

\subsection{¿Cómo desplegar un entorno en Docker Swarm?}
Para poder realizar el despliegue se ejecutará el siguiente comando dentro del nodo maestro. Para ello se necesitará disponer del \textit{docker-compose.yml} que se ha creado en el proyecto. Desde el directorio de la aplicación donde se encuentra ubicado el fichero hay que ejecutar:
\begin{verbatim}
    sudo docker stack deploy --compose-file docker-compose.yml inventarium
\end{verbatim}
Inventarium es el nombre que tendrá el deploy y del que luego se hará referencia. Puede verse el proceso de creación en la figura \ref{fig:docker-stack-deploy}.
\begin{figure}
    \centering
    \includegraphics[scale=0.5, keepaspectratio]{imagenes/complemento/docker-swarm/docker-stack-deploy.png}
    \caption{Deploy del entorno Docker en el clúster}\label{fig:docker-stack-deploy}
\end{figure}
A partir de este punto surgió una problemática y es que al analizar las máquinas que estaban ejecutándose dentro del swarm se observó que había una que no llegaba a crearse.
\\Esta máquina era la del servidor de la API con Node, el contenedor con el cual se tuvo que realizar una configuración personalizada con un Dockerfile para que funcionara.
\\Lo que ocurría era que su máquina al no estar publicada en DockerHub no podía descargarse desde ningún sitio dentro nodo esclavo teniendo por ello que publicar la imagen en Docker Hub.
\\Primero de todo había que iniciar sesión en Docker con:
\begin{verbatim}
    sudo docker login
\end{verbatim}
Luego se construirá la imagen con el Dockerfile, para poder publicarla en DockerHub el nombre de la imagen tiene que seguir la estructura \textit{nombre\_usuario/nombre\_imagen:version} ejecutando el siguiente comando en la terminal:
\begin{verbatim}
    sudo docker build -t alejomaf/inventarium_api:1.0 .
\end{verbatim}
Al generar la imagen quedaría subirla a DockerHub y eso se consigue ejecutando:
\begin{verbatim}
    sudo docker push alejomaf/inventarium_api:1.0
\end{verbatim}
Volviendo a ejecutar el deploy se obtendrá el maravilloso resultado de la figura \ref{fig:docker-swarm-final}.
\begin{figure}
    \centering
    \includegraphics[scale=0.4, keepaspectratio]{imagenes/complemento/docker-swarm/docker-FINAL.png}
    \caption{Visualización de los servicios corriendo en Docker Swarm}\label{fig:docker-swarm-final}
\end{figure}

