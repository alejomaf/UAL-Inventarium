\section{Tokenización}

La utilización de tokens durante el desarrollo de APIs ha incrementado a lo largo de los últimos años. Estos tokens permiten aislar de esa capa de seguridad de implementación extra a la que antes se estaba sometido por hacer aplicaciones que gestionasen sus propias consultas con la base de datos.
\\A medida que se iba realizando la construcción de la aplicación ya se tenía en mente esta implementación de seguridad.
\\En un principio se almacenaron los datos del usuario y la contraseña de este en caché. Cada vez que se mandaba una consulta se utilizaban los datos en caché del usuario para analizarla y proceder a un sistema de autenticación.
\\Este proceso de autenticación también iba ligado a una ``API'' podría llegar a considerarse aunque nunca llegó a ser tan potente ni bien planificada como la actual.
\\El proceso de generación de tokens desde un entorno del cliente resulta un proceso algo tedioso y que nunca hay que realizar. Por suerte, para el entorno del servidor de la API había algunas herramientas que iban a resultar de gran ayuda.

\subsection{Generación del token}
La generación de tokens siempre ha sido un proceso dudoso para el usuario intermedio, ya que, en un principio, se suele pensar que el almacenamiento de este se realiza sobre la base de datos en vez de en el propio servidor. Por suerte existe un plugin ideal que se encuentra en el Node Package Manager para esta tarea: jwt-simple.
\\Gracias a este plugin se puede realizar el codificado y decodificado de un plugin generado cada vez que un usuario iniciara sesión.
\\Dentro de las fases del codificado de este token hay que pasarle como parámetros unas fechas que será la duración que tendrán de validez estos ``simbólicos'' (traducción de token al español).
\\El proceso de codificado del token se realizará dentro del inicio de sesión. Quedando de la siguiente forma:

\begin{verbatim}
    const createToken = (usuario) => {
        let payload = {
            userId: usuario.idUsuario,
            createdAt: moment().unix(),
            expiresAt: moment().add(1, 'day').unix()
        }
        return jwt.encode(payload, process.env.TOKEN_KEY);
    };
\end{verbatim}

Esta función se ejecuta cuando se ha podido corroborar que el usuario ha iniciado sesión correctamente y sin problemas.
\\Lo que se codifica dentro de la variable TOKEN\_KEY ubicada en el archivo .env es un payload que cogerá como parámetros: el id del usuario, el momento en el que se ha logueado y el momento donde expirará su sesión.

Teniendo ya el token generado hay que, de alguna forma, hacer que cada vez que el usuario quiera hacer una consulta lo haga desde su token.

\begin{tcolorbox}
    [colback=green!5!white,colframe=green!75!black,fonttitle=\bfseries,title=¿Por qué se almacena la id del usuario en el payload?]
    La id del usuario se almacena porque cada vez que se quiere acceder a una consulta siendo clientes se podrá meter en cada una de las solicitudes ese nuevo campo. Esta acción se realizará siempre por lo que se dispondrá de los datos del usuario en cada uno de los casos de uso.
    \\Por ejemplo, si un usuario intenta acceder a una consulta la cual solo se le permite a los técnicos, gracias a poder identificar el id del usuario en base a su token podrá denegarse dicha consulta y que no se permita.
\end{tcolorbox}

\subsection{El middleware}
El middleware, como concepto, representa a todo software que se sitúa entre el software y las aplicaciones que corren sobre él. Este funciona como una capa de traducción que posibilita la comunicación y la administración de datos en aplicaciones distribuidas. En este caso la función no trabajará en si como un ``middleware'' pero si que será una capa intermedia que se irá ejecutando a medida que el usuario interactúe con la API aportando la comprobación de la validez de su token y su id de usuario.
\\Dentro del middleware se tendrá una única función que correrá en cada momento que el usuario acceda a uno de los endpoints.
\\La implementación sería la siguiente:
\begin{verbatim}
    const jwt = require("jwt-simple");
    const moment = require("moment");

    const checkToken = (req, res, next) => {
    if (!req.headers['user_token'])
        return res.json({
            error: "You must include the header"
        });

    const token = req.headers['user_token'];
    let payload = null
    try {
        payload = jwt.decode(token, process.env.TOKEN_KEY);
    } catch (err) {
        return res.json({
            error: "Invalid token"
        });
    }

    if (moment().unix() > payload.expiresAt) {
        return res.json({ error: "Expired token" });
    };

    req.userId = payload.userId;

    next();
    };
\end{verbatim}
Dentro de esta función se realizan 3 comprobaciones.

\subsubsection{Comprobación de los headers}
El token irá dentro del encabezado de las solicitudes que se hagan a la API. En concreto con la key ``user\_token''. Si no se incluye este encabezado no se podrá continuar con la comprobación por lo que se devuelve como respuesta al cliente \textit{You must include the header}.

\subsubsection{Decodificación del payload}
Luego de la comprobación de que haya ``algo'' al menos dentro del header se procede a llamar al plugin de jwt y preguntarle si tiene un payload almacenado con la clave.
\\En el caso de que no pueda identificarlo, es decir, que no sea válido, se obtendrá como respuesta en el sitio web el siguiente mensaje: \textit{Invalid token}.

\subsubsection{Verificación de la validez}
Después del segundo paso se puede haber obtenido un payload pero hay que comprobar que este no haya expirado. Es decir, hay que acceder a su variable ``expiresAt''. En el caso de que lo haya hecho devolerá un error como respuesta con el siguiente mensaje: \textit{Expired token}.
\vspace{\baselineskip}
\\Se incorporará el user id a la solicitud y con esto ya se tendría implementado el middleware.

\subsection{¿Cómo se añade el proceso de verificación a cada consulta?}
Gracias a haber hecho el código de forma estructurada y haber distinguido en servicios y rutas con solo añadir una línea al campo de rutas hará que se pueda añadir este paso de verificación intermedio.
\\Primero se importará el fichero ``middleware'' y luego se llamará en el código con la siguiente función:
\begin{verbatim}
    router.use(middleware.checkToken);
\end{verbatim}

\begin{tcolorbox}
    [colback=green!5!white,colframe=green!75!black,fonttitle=\bfseries,title=¿Cómo se inicia sesión si no se tiene un token?]
    La respuesta es muy sencilla y va en relación a la lógica del archivo de rutas. Al entrar una petición a uno de los endpoints lo que ocurre es que lo hace en forma de barrido en los ficheros. Es decir que hasta que no llegue a la línea de código de antes no se habrá hecho la comprobación.
    \\Por lo tanto, la solución es que dentro del fichero donde se realice el login y las demás acciones del usuario, se coloque esta primero en la parte de arriba, en medio la comprobación y abajo el resto de consultas que también se puedan hacer.
\end{tcolorbox}

\subsection{AuthGuard}
AuthGuard es un ``comprobador de usuarios''. Es decir, comprueba que estés logeado cogiendo tu token y mandándolo a la API, si esta no devuelve respuesta o lo hace con un \textit{null} el AuthGuard comunicaría que el usuario no está logeado.
\\Angular brinda la posibilidad de poder realizar una autenticación dentro del sistema de rutas de la aplicación. Es decir, dependiendo de a qué sitios se quiera acceder de la aplicación se pueden implementar unas medidas de seguridad u otras.
\\Esto se consigo mediante la adición del parámetro \textbf{canActivate} y \textbf{canActivateChild}:
\begin{verbatim}
    path: '', component: MainComponent, 
    canActivate: [AuthGuardService],
    canActivateChild: [AuthGuardService],
    children: [
      { path: 'dashboard', component: DashboardComponent },
      {
        path: 'add-object', component: AddObjectComponent,
        children: [
            .
            .
            .
\end{verbatim}
canActivate y canActivateChild son dos interfaces que se pueden implementar en un servicio de Angular. Estas dos interfaces van a implementar dos funciones de respectivo nombre y devolverán un booleano.
\\Este booleano, verdadero o falso, le dirá al sitio web si se puede entrar a la ruta que se esté solicitando (en caso de ser verdadero) o si no se puede (en caso de ser falso).
\\El problema que han supuesto estas interfaces es que la implementación tenía que realizarse de forma asíncrona, asincronía de consultas que se han utilizado en la API Rest pero que no se habían utilizado en el sitio web.
\\Gracias a este retraso que ha llevado poder asegurar las rutas de la aplicación se investigaron los Resolvers, los cuales son unos servicios que se pueden implementar en documentos de rutas cuando se necesite recabar una información antes de inicializar el componente de Angular. El problema era que la inicialización del componente de Angular no conllevaba la inicialización del autenticador de ruta, servicio que se ejecutaba antes.
\\Después se investigaron los \textit{Observables} y las \textit{Promises}.

\subsubsection{Observable}
\begin{itemize}
    \item Es una implementación para eventos que conlleven una carga continua a lo largo del tiempo. Por ejemplo la reproducción de un vídeo.
    \item Tiene atributos que permiten realizar nuevamente solicitudes que han devuelto un error o de las que se han perdido información.
    \item Pueden solicitar datos en varias pipelines al mismo tiempo.
\end{itemize}

\subsubsection{Promise}
\begin{itemize}
    \item Una promise puede manejar solamente un elemento.
    \item Devuelve solo un único valor.
    \item Puede manejar también errores aunque sin un control exhaustivo como un Observable.
\end{itemize}

Al observarse las diferencias entre estos dos elementos se decidió utilizar un elemento de tipo ``Promise''.
\\El segundo contatiempo ocurrido fue la creencia de que en la interfaz de canActivate y canActivateChild únicamente se podía devolver un valor booleano. Pero también pueden devolver una Promise del tipo booleana por lo que de esa forma el problema de la asincronía ya estaba resuelto.
\\El AuthGuard implementaría la siguiente función:
\begin{verbatim}
    checkLoginPromise(): Promise<boolean> {
        const promise = new Promise<boolean>((resolve, reject) => {
        this.loginS.getUser()
            .toPromise()
            .then((res: any) => {
                // Success
                this.usuario = res;
                resolve(true);
            },
            err => {
                // Error
                this.logout();
                reject(false);
            }
            );
        });
        return promise;
    }
\end{verbatim}
Se puede comprobar que los dos valores que puede devolver la promesa son verdadero o falso. En caso de que sea verdadero el usuario quedará cargado en el sitio web para poder gestionar algunos estilos que requieren sus datos dentro de ella.
\\Y por último están los dos métodos canActivate y canActivateChild:
\begin{verbatim}
    canActivateChild(): Promise<boolean> {
        return this.checkLoginPromise();
    }

    canActivate(): Promise<boolean> {
        return this.checkLoginPromise();
    }
\end{verbatim}



