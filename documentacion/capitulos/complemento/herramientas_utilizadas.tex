\chapter{Herramientas utilizadas}

En este capítulo hablaré sobre las herramientas utilizadas durante el desarrollo del proyecto. Algunas de ellas ya las habré presentado en el anterior proyecto pero no serán demasiadas para no volver tan extenso el documento.
\\En la sección de hardware, eso sí, disponemos de los mismos dispositivos que antes así que no va a haber variación.

\section{Hardware}

Dentro del apartado de hardware disponemos de dos ordenadores. Su procesador no conlleva relevancia en el desarrollo de la aplicación debido a la utilización de servicios en la nube.

\subsection{Torre de PC}

La cual contiene como procesador un Xeon E5-2620 V3. 16GB de RAM DDR3. Una tarjeta gráfica RTX 570 de 4GB DDR5. Tiene 256GB de memoria SSD y 1TB de memoria HDD.

\subsection{Un portátil}

Es un MacBook Air M1 de 2020 con 8GB de RAM y 256GB de almacenamiento.

\section{Software}

\subsection{Entorno de desarrollo}

Nuestro entorno de desarrollo y desde donde haremos casi absolutamente todo será desde Visual Studio Code. Este es un editor de código desarrollado por Microsoft que soporta varias distribuciones de sistemas operativos, entre ellas: Windows, Mac Os y Ubuntu. 
\\Una de las características de esta herramienta que la hacen la predilecta de varios desarrolladores es el gran soporte que tiene por parte de la comunidad. Tiene un mercado de plugins bastante grande que apoya la creación continua de código para todos los desarrolladores.
\\Dispone de integraciones con Git, resaltado en errores de sintaxis, finalización de código y hasta conexión remota a otros entornos de trabajo mediante SSH.
\\Otra enorme ventaja que presenta es el consumo de memoria que tiene, bastante pequeño. Es un programa para ordenadores de todos los tamaños y precios, un software gratuito y una herramienta increíblemente potente al alcance de todos.

\subsection{Redacción del documento}

Estas líneas están siendo escritas ahora mismo desde LaTeX. LaTeX es un sistema de composición de textos que está formado mayoritariamente por órdenes construidas a partir de comandos TeX. En un principio no estaba seguro de qué herramienta utilizar, ya que la posición de varios profesores respecto a esta herramienta era bastante férrea pero Word siempre había ido agarrado a mi mano desde comienzos del instituto.
\\Luego de pasar de Word a LaTeX y de LaTeX a Word bastantes veces no fue hasta que mi profesora Rosa, en una de mis visitas matinales a su despacho me dijo: Yo hice mi TFG en LaTeX.
\\No me lo podía creer y al comprobar la fecha de publicación de este programa de procesado de textos me sorprendí al ver que su lanzamiento oficial fue en 1980. ``Si el programa ha durado tanto es que algo de importante tendrá'' pensé. Y aquí me hallo redactando este documento con un programa que facilita el control de versiones de Git de una manera asombrosa. Facilita también los procesos de documentación y disposición de las diferentes subsecciones. Y, lo que más me gusta sin lugar a dudas, que puedo realizar una separación de cada capítulo por documentos separados y es que a mí, el tener las cosas descompuestas, me puede.

\subsection{Google Cloud}
Volveremos a usar Google Cloud pero esta vez no para el deploy de nuestra página web. Ya que ya lo hemos hecho. Sino que será para la generación de un entorno red donde desplegaremos una o dos máquinas virtuales más. Esto lo haremos con el objetivo de generar una unión entre ellas y poder utilizar una de las herramientas que nos incorpora Docker.

\subsection{Docker Swarm}

Docker Swarm nos permite tener varios contenedores Docker interconectados entre sí en diferentes máquinas virtuales o físicas.
\\Es decir, Docker Swarm nos permite la gestión de un clúster de servidores Docker. Además nos aporta herramientas para poder gestionar estos como si se tratase de la gestión de un simple contenedor.
