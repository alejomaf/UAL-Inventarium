\chapter{Introducción}

Son varios los caminos que han llevado al desarrollo de este proyecto. En este capítulo tendré la oportunidad de poder hablar de cada uno de ellos. Desde empezar a estudiar ingeniería informática hasta la mentalidad que he ido adquiriendo a lo largo de estos años gracias a ella.

\section{Motivaciones}

La motivación principal ha sido la ingeniería informática y es que sin ella no sería quién soy ahora. Para bien y para mal, ha ido siendo una evolución de pequeños pasos empezando en primero de carrera hasta llegar a saber una pizca de todo este mundo de tecnología ya acabando cuarto. Todo empezó desde pequeño, desde que vi esos monitores gigantes, los disquetes y el Age of Empires 1 a la edad de cuatro años. 
\\Mi madre tuvo un novio, Martín se llamaba, que nos trajo el mundo de la tecnología a la casa. Ella también era otra fan tecnológica, pues al haberme tenido con tan solo 17 años siempre tuvo esa mentalidad más abierta con todo el tema de nuevas herramientas. Creo que esto me acabó ayudando bastante.
\\Luego de esta pequeña época vino una un poco más oscura que consistía en una vida de cero tecnología. Influía bastante el aspecto de vivir en Argentina y que el precio de los aranceles fuera bastante alto influyendo en los productos tecnológicos en su mayoría. Pasar al lado de los escaparates y ver esas pantallas brillantes con juegos, películas, canciones\ldots era un martirio para mí.
\\Otro aspecto muy importante y el cual también he de agradecer fue la concesión que hizo la Junta de Andalucía allá por 2009 de aquellos pequeños ordenadores verdes. Esos sí que fueron gasolina para mis aspiraciones y para terminar hoy aquí, dentro del grado de Ingeniería Informática de la Universidad de Almería.
\\\\Mi segunda motivación principal para el desarrollo de este proyecto fue la beca extracurricular que publicó el Departamento de Informática de la Universidad de Almería, cuyo director en el momento de la concesión y en la fecha actual de redacción de este documento es Juan Francisco Sanjuan Estrada, el tutor de este proyecto.
\\Gracias a ella me sentí motivado para poder afianzar la actividad desempeñada en aquel pequeño trayecto de tiempo, unos seis meses, a la redacción de este trabajo.
\\El trabajo no se basa en lo que exactamente realicé en aquella beca, sino en una refactorización/transformación de aquel proyecto para poder convertirla en algo ``fresco'' podríamos llamarlo.
\\\\Y aquí pasamos al tercer y último punto de este apartado. La transformación del proyecto, el aplicar el concepto de ingeniería para poder transformar algo que era un software cerrado, con poca modularidad, difícil de entender y de ampliar en algo que merezca la pena tener. Un software que combine tecnologías y metodologías de hoy en día. Un ejemplo a seguir.

\section{Objetivos}

El objetivo principal de este proyecto es la creación de un sistema de gestión de inventario y creación de préstamos para el Departamento de Informática de la Universidad de Almería.
\\Este sitio web tiene que cumplir con unos requisitos principales, que son:

\begin{itemize}
    \item Llevar un registro del inventario del Departamento de Informática ubicado en el edificio Científico Técnico III
    \item Que el estudiantado y el personal docente e investigador realicen solicitudes de préstamos para los distintos elementos ofertados dentro de la página
    \item Que los técnicos de servicio puedan gestionar estas solicitudes más el seguimiento del inventario dentro del edificio
\end{itemize}
Por estos puntos entendemos que la motivación principal de la herramienta es la de gestionar y organizar préstamos.
\\Luego de los requisitos principales que había que cumplir se nos presentaban los secundarios.

\begin{itemize}
    \item La página web tenía que disponer de un modo adaptativo para las versiones móviles
    \item Había que incorporar la posibilidad de realizar un importado de datos gracias al procesamiento de una tabla de datos que es lo que antiguamente utilizaban los técnicos del departamento
    \item Había una serie de datos que tenían que almacenar los objetos añadidos a este sistema los cuales podían ser de tres tipos: inventarios, fungibles o kits
    \item Se permitiría ver un seguimiento de los préstamos realizados sobre los objetos.
\end{itemize}
En base a estos objetivos principales y secundarios realizaremos la construcción de la aplicación.

\section{Planificación}

La planificación del proyecto se dividirá en cinco grandes grupos: reuniones iniciales con los clientes, planificación y elaboración del desarrollo del proyecto, preparación del entorno de trabajo, construcción de la aplicación, testeo y comprobación de la aplicación y comprobación de errores.

\subsection{Reuniones iniciales con los clientes}

A pesar de basarse en la reconstrucción de una aplicación las reuniones con los clientes son iniciales. Estas se hicieron en un principio del desarrollo y se mantuvieron unas pocas más a lo largo del mismo. Terminando añadiendo nuevas funcionalidades o descripciones de los productos.

\subsection{Planificación y elaboración del desarrollo del proyecto}