\chapter{Deploy del sitio web}
Al terminar la construcción de la página lo que se hará será preparar los archivos para añadirlos al servidor web.
\\Uno de los objetivos de la preparación del deploy del sitio es que estuviera todo en un archivo docker-compose. Esto en vistas a una mejor configuración y a poder asegurar el entorno antes de llevarlo a producción.
\\Por lo que el fichero docker-compose se tendría que encargar de levantar los siguientes elementos:

\begin{itemize}
    \item La base de datos de MariaDB.
    \item El servidor Node que manejara la API.
    \item El servicio de PHPMyAdmin para manejar la base de datos ante cualquier problemática.
    \item El servidor Web con el Deploy de Angular que iría con Nginx.
\end{itemize}

Más adelante en un futuro se añadiría otro bloque más que sería el gestor de copias de seguridad de la base de datos.

\section{Levantar el servidor para la API}
Para poder levantar el servidor con la API era necesario sí o sí realizar una configuración de una imagen docker Node para poder inicializar el Node Package Manager \cite{docker-nodejs}.
\\También, con el objetivo de ahorrar intermediarios, se quería poder generar la imagen Docker desde Docker Compose.
\\Un fichero Dockerfile sirve para generar una imagen docker (no un contenedor) que se podrá ir manipulando para que tenga la configuración que sea necesaria para que provea de los servicios necesarios para UAL Inventarium.
\\La estructura de un archivo Dockerfile es la siguiente:
\begin{verbatim}
    FROM node:16.4
\end{verbatim}
Primero se define la imagen desde la que se partirá. Este es un proceso obligatorio. Se generará la imagen para la API desde la versión de node 16.4 que es con la que se ha llevado a cabo el desarrollo.
\begin{verbatim}

    WORKDIR /app
    COPY package*.json ./
    RUN npm install
    COPY . .

    
\end{verbatim}
El segundo paso será definir la dirección del directorio de trabajo. Esta dirección \textit{/app} será donde se ubicará el directorio raíz de la API.
\\Luego se copiarán todos los archivos que empiecen por \textit{package} que en este caso son dos: \textit{package.json} y \textit{package-lock.json}. Dentro de estos ficheros se encuentra toda el versionado y la configuración de los plugins que se utilizarán en el sistema.
\\Luego de copiar la configuración del versionado se procederá a iniciar Node Package Manager. Este, en base a los paquetes que se han declarado anteriormente, será el encargado de generar todos los directorios que necesita Node para funcionar.
\\Después de haber instalado NPM el siguiente paso es copiar todo el directorio raíz de la API dentro de la imagen que está siendo creada, eso se hará con \textit{``COPY . .''}.
\begin{tcolorbox}
    [colback=green!5!white,colframe=green!75!black,fonttitle=\bfseries,title=¿Por qué no se copia directamente el directorio raíz con los plugins y demás?]
    Lo que ocurre cuando haces una copia exacta es que tu sistema ha adaptado las librerías y alguna configuración extras a tu entorno. Por lo que luego al contenerizarlo da error. Lo bueno que permite Docker es que esté donde esté el sistema que se vaya a utilizar, mientras esté en un contenedor, esté será el mismo para todos los usuarios.
\end{tcolorbox}
\begin{verbatim}
    EXPOSE 3000
    CMD ["npm", "run", "dev"]
\end{verbatim}
Lo siguiente sería abrir el puerto 3000 para que el usuario pueda mandar solicitudes a la API y ejecutar el comando \textit{npm run dev} para que esta empiece a funcionar.
\\Junto a este Dockerfile también hay que generar un \textit{.dockerignore} que al igual que \textit{.gitignore} sirve para que en el momento de la creación de la imagen que se hace con el Dockerfile el sistema Docker no coja ni los ficheros ni directorios que se encuentren declarados dentro del documento.
\\El archivo contiene los siguientes campos de texto:
\begin{verbatim}
    node_modules
    Dockerfile
    .git
\end{verbatim}
Se ignorará el directorio \textit{node\_modules} que es donde se descargan los paquetes. El \textit{Dockerfile} y el \textit{.git} es para que la imagen no tenga contenido extra que no se vaya a utilizar.
\vspace{\baselineskip}
\\Con todo esto ya definido se pasaría a añadir el apartado de la API en el Docker Compose.
\begin{verbatim}
    api:
        build:
            context: ../API
            dockerfile: ../API/Dockerfile
\end{verbatim}
Dentro de la sección \textit{build} se define el sitio donde Docker Compose generará la imagen que se vaya a utilizar para el contenedor. En el apartado \textit{context} se define el directorio que será la raíz de la generación y en el apartado \textit{dockerfile} como su propio nombre indica se seleccionará el archivo Dockerfile a utilizar.
\begin{verbatim}
        image: inventarium_api:1.0
        restart: always
\end{verbatim}
Aquí se define el nombre de la imagen que se estça creando y \textit{restart: always} servirá para que en el momento en el que el contenedor se pare este se vuelva a iniciar.
\begin{tcolorbox}
    [colback=green!5!white,colframe=green!75!black,fonttitle=\bfseries,title=¿Cuándo se para un contenedor?]
    Un contenedor puede dejar de funcionar por varias razones pero, la principal, es porque se ha quedado sin tareas que realizar. En el caso de que ocurriese eso en el sistema, significaría que la API ha dado un error y se ha parado. Al no querer que esta deje de funcionar se programa para que se vuelva a encender.
\end{tcolorbox}
\begin{verbatim}
        container_name: api
        ports:
            - "3000:3000"
        volumes:
            - ../API:/app
        networks:
            - default
\end{verbatim}
Por último se define el nombre del contenedor, para que redirija del puerto 3000 a el puerto 3000 del sistema, que realice un copiado de archivos del contenido de la carpeta API dentro de la raíz del sistema en caso de que haya nuevos archivos para subir y por último que esté funcionando sobre la red con el nombre \textit{default}.

\section{Levantar el servidor Web}
Han surgido dos problemas principales al intentar realizar esto:
\begin{enumerate}
    \item Configurar las rutas de la aplicación desde el servidor web.
    \item Configurar un proxy reverso desde el servidor web.
\end{enumerate}
El servidor web que se utilizó fue \textbf{Nginx}. La configuración inicial fue bastante rápida. Quedando el fichero \textit{docker-compose.yml} así en un principio:
\begin{verbatim}
    client:
        image: nginx:latest
        ports:
            - 80:80
        volumes:
            - ../inventarium/dist/inventarium:/usr/share/nginx/html
\end{verbatim}
Se importa la última imagen de nginx, se redirige el puerto 80 al puerto 80 y por último se importa el contenido del sitio web dentro de el directorio \textit{/usr/share/nginx/html} de nginx.
\\En un principio parecía que estaba todo bien pero al acceder al sitio web ocurrieron dos cosas: que no llegaban las solicitudes a la API y que tampoco podía accederse al archivo de rutas más allá que la raíz de Angular. Es decir, no podía accederse a la ruta \textit{/group-of-objects} por ejemplo.
\\Con \textit{ng build} no se había importado el proxy que se había creado en el sitio web y tampoco funcionaba la configuración de rutas, aunque, sí se podía navegar dentro de la aplicación.
\\Esto se debía a la configuración de rutas que estaban configuradas en nginx. Rutas que se modificaban desde \textit{/etc/nginx/conf.d/default.conf}.
\\Por lo tanto se creó un archivo llamado \textit{default.conf} y se añadió una línea más en la sección de \textit{volumes} del fichero del Compose. Esta línea era:
\begin{verbatim}
            - ./web/default.conf:/etc/nginx/conf.d/default.conf
\end{verbatim}
Que ayudaba a configurar las rutas en default.conf.

\subsection{Configuración del ruteo de la web}
El objetivo de configurar las rutas de la web es que todas las direcciones apuntásen al mismo archivo \textit{index.html}. Para eso se añadieron dentro del fichero las siguientes líneas \cite{nginx-routing}:
\begin{verbatim}
    location / {
      root /usr/share/nginx/html;
      try_files $uri $uri/ /index.html;
    }
\end{verbatim}
Esta configuración define las rutas entrantes a la dirección raíz \textit{``/''}, en el caso de que sean allí apuntará al directorio \textit{html} que ha sido indicado anteriormente y si no es el caso lo hará sobre \textit{try\_files \$uri \$uri/ /index.html;}. Sobre index.html. Es decir, lo mismo que si apuntara a la raíz del servidor.

\subsection{Configuración del proxy inverso de la web}
Como se explicó en la sección \ref{sec:4_web}, para poder conseguir que la aplicación se comunique con la API esta tiene que disponer de un proxy que rediriga unas determinadas rutas dentro del puerto 80 a otras del puerto 3000.
\\Para poder hacer esto en Nginx hay que añadir las siguientes líneas de código en el archivo anterior \cite{proxy-reverse}:
\begin{verbatim}
    location /api/users/login {
        proxy_pass http://api:3000;
        proxy_pass_request_headers on;
    }
\end{verbatim}
Sigue casi la misma estructura que en el apartado anterior siendo \textit{proxy\_pass} la dirección donde se reenviarán las peticiones. El dominio donde las reenvia se llama \textit{api}, esto es debido a que dentro del entorno que genera Docker Compose pueden llamarse a las máquinas creadas en base a su referenciación.
\\La siguiente línea sirve para poder reenviar el encabezado de las solicitudes para que en la ampliación del proyecto pueda generarse un sistema de tokens con el objetivo de querer aumentar la seguridad.
\\Con esto ya estarí creado el servidor web donde se localizaría la aplicación.
