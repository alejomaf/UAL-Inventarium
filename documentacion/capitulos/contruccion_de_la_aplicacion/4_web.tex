\section{Creación de UAL Inventarium}
Este es la sección donde se juntan todos los conocimientos y herramientas que hemos ido exponiendo a lo largo de este documentos y los cohesionamos para crear UAL Inventarium.
\\UAL Inventarium o Inventarium es una herramienta web diseñada para la gestión del inventario del Departamento de Informática de la Universidad de Almería.
\\La página web está hecha en Angular 12. Angular es una plataforma de desarrollo compuesta por un framework y librerías. Angular nos brinda todas las herramientas necesarias para la creación de un sitio web.
\\Para poder desplegar un entorno donde trabajar en Angular primero tenemos que instalarlo en nuestro Node Package Manager (NPM) con el siguiente comando:
\begin{verbatim}
    npm i -g @angular/cli
\end{verbatim}
Al hacer esto se nos desplegará nuestro entorno de desarrollo para Angular. El directorio \textbf{node\_modules} es donde se almacena el Framework de Angular, el CLI y los distintos componentes que vayamos instalando con el NPM.
\begin{tcolorbox}
    [colback=green!5!white,colframe=green!75!black,fonttitle=\bfseries,title=¿Qué diferencias hay entre Angular CLI y Angular Framework?]
    Angular CLI es la Command Line Interface la cual permite poder desplegar proyectos Angular, añadir componentes, servicios o directivas desde una línea de comandos. Angular CLI se encarga de la gestión de las distintas posibilidades que nos puede ofrecer el framework de Angular.
\end{tcolorbox}
Los ficheros que se nos despliegan sobre el directorio raíz al crear un proyecto Angular son:
\begin{itemize}
    \item
\end{itemize}
