\section{Introducción a Docker y Docker Compose}
Al finalizar cada apartado se explicará cómo se realizó su adición dentro de nuestro entorno que despliega Docker Compose. Pero, ¿cómo funciona Docker Compose? O mejor dicho ¿para qué vamos a necesitar docker?

\subsection{¿Para qué sirve Docker?}
Gracias a Docker en vez de máquinas virtuales podemos utilizar contenedores. Los contenedores presentan diferentes ventajas:

\begin{itemize}
    \item El entorno de pruebas donde ejecutaremos los distintos componentes de la aplicación son idénticos al de un servidor.
    \item Obtenemos mayor modularidad. Gracias a esto tenemos un entorno ideal donde poder trabajar con microservicios.
    \item Podemos ejecutar nuestra aplicación en un entorno que sabemos que en caso de fallo lo volvemos a levantar.
\end{itemize}

Docker es un avance para la informática. Nos ayuda a ahorrar recursos en nuestro ordenador y poder dedicar el tiempo invertido en la configuración de entornos de servidor a otras actividades.

\subsection{Docker Compose: el SimCity de los entornos}

¿Por qué SimCity? Al igual que en el famoso juego SimCity levantávamos ciudades en cuestión de segundos gracias a Docker Compose levantamos entornos en el mismo tiempo.
\\Docker Compose nos ayuda a definir un entorno de contenedores con la posibilidad de poder conectarlos entre ellos. También nos ofrece la posibilidad de generar volúmenes de almacenamiento donde ir almacenando la información generada dentro de esos volúmenes. Lo hace todo menos la comida, ojalá.
\\Docker Compose funciona sobre un archivo llamado docker-compose.yml dentro del cual iremos añadiendo los componentes que querramos configurar. Todo esto lo iremos viendo a lo largo de las siguientes secciones.