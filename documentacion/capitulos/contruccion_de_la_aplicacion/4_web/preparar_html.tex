\subsubsection{Preparar el archivo HTML}
El fichero HTML será el siguiente:
\begin{verbatim}
    <div class="row d-flex-inline justify-content-center">
    <!--Contenido-->
    <div *ngFor="let go of group_of_objects" style="width: fit-content;">
        <app-group-of-object>
            <img imagen class="card-img-top rounded-circle" 
            style="cursor:pointer" src="http://api/images/{{go.imagen}}.jpg">
            <h5 nombre [routerLink]="['/group-of-object', 
            go.idGrupoObjetos]" style="cursor:pointer"
            class="card-title text-dark text-center">
                {{go.nombre}}
            </h5>
            <span cantidad>{{go.cantidad}}</span>
            <span marca>{{go.marca}}</span>
            <span modelo>{{go.modelo}}</span>
            <span cantidadDisponible>{{go.cantidadDisponible}}</span>
            <span *ngIf="go.tipo==0" tipo>Inventario</span>
            <span *ngIf="go.tipo==1" tipo>Fungible</span>
            <span *ngIf="go.tipo==2" tipo>Kit</span>
        </app-group-of-object>
    </div>
</div>
\end{verbatim}
El componente es bastante sencillo. Primero se define un contenedor donde se irán insertando los componentes que haya devuelto la petición mandada anteriormente en el constructor.
\\Ahora se definirá otro contenedor donde se iterará sobre el array de groupos de objetos asignándolo a una variable auxiliar \textit{go}. Esto se hará utilizando \textit{*ngFor}.
\\Luego de iterar se llamará al componente hijo que será el que se irá generando por cada grupo de objeto que cargue.
\\Dentro del componente hijo se definirán los componentes HTML que se quieran pasar. Para poder definir esto basta con añadirle un nombre que después se referenciará en el otro componente.
\\Puede verse al final del código que se define un atributo dentro del componente HTML \textit{span} llamado \textit{*ngIf}. Este atributo es un condicional, en caso de que sea \textit{true} se cargará el componente. Si es \textit{false} no lo hará.
\\Ahora se definirá componente hijo que es al que se está llamando:
\begin{verbatim}
    <div class="card border rounded p-3 m-2" 
    style="width: 22rem;background-color:#FDF7FF;">
    <div style="width: 100%; height: 230px;">
        <ng-content select="[imagen]"></ng-content>
    </div>
    <div style="width: 100%;" class="card-body list-group-item-dark border">
        <ng-content select="[nombre]"></ng-content>
    </div>
    <ul class="list-group list-group-flush">
        <li class="list-group-item bg-light">
            <b>Marca</b>
            <a>:<ng-content select="[marca]"></ng-content></a>
        </li>
        <li class="list-group-item bg-light">
            <b>Modelo</b>
            <a>:<ng-content select="[modelo]"></ng-content></a>
        </li>
        <li class="list-group-item bg-light">
            <b>Cantidad</b>
            <a>:<ng-content select="[cantidad]"></ng-content></a>
        </li>
        <li class="list-group-item bg-light">
            <b>Cantidad disponible</b>
            <a>:<ng-content select="[cantidadDisponible]"></ng-content></a>
        </li>
    </ul>
    <li class="list-group-item list-group-item-dark 
    font-weight-bold text-center">
        <ng-content select="[tipo]"></ng-content>
    </li>
    <ng-content select="[botones]"></ng-content>
</div>
\end{verbatim}
Este es el modelado que tiene el objeto. El componente HTML llamado \textit{ng-content} será el encargado de tomar los objetos que le está pasando el componente padre. Estos los referencia con el atributo \textit{select} y le indica el nombre del tipo de componente que quiere coger.
\\Para poder ir revisando cómo quedan los componentes se utilizará una funcionalidad que incorpora Angular. Dentro del archivo \textit{package.json} NPM define una serie de scripts que pueden utilizarse. Uno de ellos es el siguiente:
\begin{verbatim}
    "start": "ng serve"
\end{verbatim}
Para poder ejecutar este script se ejecutará el siguiente comando:
\begin{verbatim}
    npm run start
\end{verbatim}
Que sería lo mismo que utilizar:
\begin{verbatim}
    ng serve
\end{verbatim}
Esto desplegará un servidor de la aplicación de Angular en el puerto 4200. Una de las ventajas que ofrece esto es una compilación continua del proyecto. Es decir, a medida que se vaya realizando la contruccion de la aplicación, la generación de componentes y la creación de rutas la web se irá actualizando en cada guardado y será de gran utilidad poder ir viendo estos cambios al momento.