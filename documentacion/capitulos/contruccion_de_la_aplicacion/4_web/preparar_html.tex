\subsubsection{Preparar nuestro archivo HTML}
Nuestro fichero HTML será el siguiente:
\begin{verbatim}
    <div class="row d-flex-inline justify-content-center">
    <!--Contenido-->
    <div *ngFor="let go of group_of_objects" style="width: fit-content;">
        <app-group-of-object>
            <img imagen class="card-img-top rounded-circle" 
            style="cursor:pointer" src="http://api/images/{{go.imagen}}.jpg">
            <h5 nombre [routerLink]="['/group-of-object', 
            go.idGrupoObjetos]" style="cursor:pointer"
            class="card-title text-dark text-center">
                {{go.nombre}}
            </h5>
            <span cantidad>{{go.cantidad}}</span>
            <span marca>{{go.marca}}</span>
            <span modelo>{{go.modelo}}</span>
            <span cantidadDisponible>{{go.cantidadDisponible}}</span>
            <span *ngIf="go.tipo==0" tipo>Inventario</span>
            <span *ngIf="go.tipo==1" tipo>Fungible</span>
            <span *ngIf="go.tipo==2" tipo>Kit</span>
        </app-group-of-object>
    </div>
</div>
\end{verbatim}
El componente es bastante sencillo. Primero definimos un contenedor donde iremos insertando tantos componentes nos haya devuelto la petición mandada anteriormente.
\\Ahora definimos otro contenedor donde iteraremos sobre nuestro array de groupo de objetos asignandolo a una variable auxiliar \textit{go}. Esto lo haremos utilizando \textit{*ngFor}.
\\Luego de iterar llamaremos a nuestro componente hijo que será el que iremos generando por cada grupo de objeto que cargue. Este componente lo veremos ahora después.
\\Dentro del componente hijo definiremos los componentes HTML que le queramos pasar. Para poder definir esto basta con añadirle un nombre que después referenciaremos en el otro componente.
\\Podemos ver al final del código que definimos un atributo dentro de nuestro componente HTML \textit{span} llamado \textit{*ngIf}. Este atributo es un condicional, en caso de que sea \textit{true} se cargará el componente. Si es \textit{false} no lo hará.
\\Ahora definiremos nuestro componente hijo que es al que estamos llamando:
\begin{verbatim}
    <div class="card border rounded p-3 m-2" 
    style="width: 22rem;background-color:#FDF7FF;">
    <div style="width: 100%; height: 230px;">
        <ng-content select="[imagen]"></ng-content>
    </div>
    <div style="width: 100%;" class="card-body list-group-item-dark border">
        <ng-content select="[nombre]"></ng-content>
    </div>
    <ul class="list-group list-group-flush">
        <li class="list-group-item bg-light">
            <b>Marca</b>
            <a>:<ng-content select="[marca]"></ng-content></a>
        </li>
        <li class="list-group-item bg-light">
            <b>Modelo</b>
            <a>:<ng-content select="[modelo]"></ng-content></a>
        </li>
        <li class="list-group-item bg-light">
            <b>Cantidad</b>
            <a>:<ng-content select="[cantidad]"></ng-content></a>
        </li>
        <li class="list-group-item bg-light">
            <b>Cantidad disponible</b>
            <a>:<ng-content select="[cantidadDisponible]"></ng-content></a>
        </li>
    </ul>
    <li class="list-group-item list-group-item-dark 
    font-weight-bold text-center">
        <ng-content select="[tipo]"></ng-content>
    </li>
    <ng-content select="[botones]"></ng-content>
</div>
\end{verbatim}
Este es el modelado que tiene nuestro objeto. El componente HTML llamado \textit{ng-content} será el encargado de tomar los objetos que le está pasando el componente padre. Estos los referencia con el atributo \textit{select} y le indica el nombre del tipo de componente que quiere coger.
\\Para poder ir revisando cómo quedan nuestros componentes utilizaremos una funcionalidad que nos incorpora Angular. Dentro de nuestro archivo \textit{package.json} NPM nos define una serie de scripts que podemos utilizar. Uno de ellos es el siguiente:
\begin{verbatim}
    "start": "ng serve"
\end{verbatim}
Para poder ejecutar este script ejecutaríamos el siguiente comando:
\begin{verbatim}
    npm run start
\end{verbatim}
Que sería lo mismo que utilizar:
\begin{verbatim}
    ng serve
\end{verbatim}
Esto nos despliega un servidor de la aplicación de Angular en el puerto 4200. Una de las ventajas que nos ofrece esto es una compilación continua del proyecto. Es decir, a medida que vayamos realizando la contruccion de la aplicación, la generación de componentes y la creación de rutas la web se irá actualizando en cada guardado y será de gran utilidad poder ir viendo estos cambios al momento.