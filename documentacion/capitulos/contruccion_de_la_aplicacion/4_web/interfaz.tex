\subsubsection{Interfaz}
La interfaz sirve para definir los elementos con los que vamos a trabajar. Estos elementos corresponden a los que tenemos en nuestra base de datos.
\\Para poder generar una interfaz escribimos lo siguiente en la terminal dentro de nuestro proyecto:
\begin{verbatim}
    ng g i directorio_y_nombre_de_la_interfaz
\end{verbatim}
Un ejemplo de interfaz sería por ejemplo la de un objeto del tipo \textit{Configuración}:
\begin{verbatim}
    export interface Configuracion {
        idConfiguracion: number,
        ip: string,
        mac: string,
        boca: string,
        armario: string,
        usuario: string,
        contrasena: string,
        Objeto_idObjeto?: number
    }
\end{verbatim}
Estos elementos también nos ayudarán a procesar las respuestas que nos llegarán desde la API y poder manipularlos en nuestros componentes sin problemas.
\vspace{\baselineskip}
\\Ya sabemos qué tipos de componentes conforman un proyecto Angular. Ahora metemos de lleno las manos en la masa.