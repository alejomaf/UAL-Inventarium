\subsubsection{Componente}
Los componentes son las estructuras principales de construcción que hay en Angular \cite{angular-components}. Para poder generarlos se utiliza el siguiente comando:
\begin{verbatim}
    ng g c direccion_y_nombre_del_componente
\end{verbatim}
Cuando se ejecute se generará un nuevo directorio con el nombre del componente. Dentro de él se habrán creado cuatro ficheros diferentes:
\begin{itemize}
    \item \textbf{componente.html}: Aquí irá ubicado el diseño html que tendrá el componente.
    \item \textbf{componente.css}: Este documento de estilos se aplicará unicamente al componente.
    \item \textbf{componente.ts}: En el fichero TypeScript está la lógica del componente y cualquier tipo de procesado de datos que haya que realizar.
    \item \textbf{componente.spects.ts}: Este será el fichero de pruebas unitarias para el componente. Durante el desarrollo del proyecto no se ha utilizado ya que el testing del proyecto se ha hecho de otra forma.
\end{itemize}