\subsubsection{Componente}
Los componentes son las estructuras principales de construcción que tenemos en Angular. Para poder generarlos utilizamos el siguiente comando:
\begin{verbatim}
    ng g c direccion_y_nombre_del_componente
\end{verbatim}
Cuando lo ejecutemos generará un nuevo directorio con el nombre del componente. Dentro de él se habrán creado cuatro ficheros diferentes:
\begin{itemize}
    \item \textbf{componente.html}: Aquí irá ubicado el diseño html que tendrá el componente.
    \item \textbf{componente.css}: Este documento de estilos se aplicará unicamente al componente.
    \item \textbf{componente.ts}: En nuestro fichero TypeScript tenemos la lógica del componente y cualquier tipo de procesado de datos que haya que realizar.
    \item \textbf{componente.spects.ts}: Este será nuestro fichero de pruebas unitarias para el componente. Durante el desarrollo del proyecto no lo he utilizado ya que el testing del proyecto lo he hecho de otra forma.
\end{itemize}