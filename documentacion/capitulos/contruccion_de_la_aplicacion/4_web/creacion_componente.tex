\subsection{Creación de componentes}
Sabiendo cómo crear un componente se procederá a explicar un poco de forma más detallada su funcionamiento.
\\Como ejemplo se hablará del componente \textit{groups-of-objects} que es donde se visualizan todos los grupos de objetos.

\subsubsection{Preparar el apartado TypeScript}
Se declaran las variables que se necesitarán, en este caso solo es:
\begin{verbatim}
    group_of_objects: GrupoObjetos[] = [];
\end{verbatim}
Dentro del código también se inicializa el array del grupo de objetos. Como se puede ver al declarar la variable se ha llamado a su interfaz para que su utilización sea mucho más cómoda.
\\Para poder cargar los grupos de objetos que hay definidos en Inventarium hay que importar su servicio:
\begin{verbatim}
    constructor(private group_of_objects_service: GroupOfObjectsService)
\end{verbatim}
Y para que este cargue los respectivos grupos de objetos, dentro del constructor habrá que llamar al método para que lo haga:
\begin{verbatim}
    this.group_of_objects_service.getGroupOfObjects().subscribe(
    (data : any) => { 
        this.group_of_objects = res.data;
    },err => console.log('Error', err));
\end{verbatim}
El método \textit{getGroupOfObjects()} es el método que se ha definido anteriormente, el \textit{.subscribe} es para poder realizar la consulta. Dentro de él puede decidirse cómo manipular los elementos que devuelva la petición.
\\Este elemento es un objeto del \textit{json} pero Angular lo interpreta perfectamente. Objeto que puede tener uno de estos dos atributos, \textit{data} que es un array de grupo de objetos, significando que la consulta no ha tenido errores y \textit{err} que devuelve una cadena de carácteres en las que sale el tipo de error que ha ocurrido.
\\Con esto ya estaría el objeto cargado, pero ahora hay que mostrarlo.