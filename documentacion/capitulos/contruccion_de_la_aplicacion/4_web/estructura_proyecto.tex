\subsection{Estructura de un proyecto Angular}
Para poder desplegar un entorno donde trabajar en Angular primero tenemos que instalarlo en nuestro Node Package Manager (NPM) con el siguiente comando:
\begin{verbatim}
    npm i -g @angular/cli
\end{verbatim}
Al hacer esto se nos desplegará nuestro entorno de desarrollo para Angular. El directorio \textbf{node\_modules} es donde se almacena el Framework de Angular, el CLI y los distintos componentes que vayamos instalando con el NPM.
\begin{tcolorbox}
    [colback=green!5!white,colframe=green!75!black,fonttitle=\bfseries,title=¿Qué diferencias hay entre Angular CLI y Angular Framework?]
    Angular CLI es la Command Line Interface la cual permite poder crear proyectos Angular, añadir componentes, servicios o directivas desde una línea de comandos. Angular CLI se encarga de la gestión de las distintas posibilidades que nos puede ofrecer el framework de Angular.
\end{tcolorbox}
Los ficheros que se nos despliegan sobre el directorio raíz al crear un proyecto Angular son:
\begin{itemize}
    \item \textbf{.editorconfig}: Un archivo de configuración para editores de código.
    \item \textbf{README.MD}: Archivo de texto que procesa GitHub en sus repositorios. El contenido inicial del fichero en el momento de la creación del proyecto trata sobre documentación acerca del Framework.
    \item \textbf{angular.json}: Esta es la configuración predeterminada que nos aporta el CLI de Angular para poder construir la aplicación, generar el servicio y testear los diferentes componentes.
    \item \textbf{package.json}: Este fichero se encarga de manejar las dependencias de NPM, precisamente las que están habilitadas dentro del espacio de trabajo.
    \item \textbf{package-lock.json}: Nos aporta información del versionado de los distintos paquetes que están en node\_modules.
    \item \textbf{tsconfig.json}: Esta es la configuración básica de TypeScript para el proyecto.
    \item \textbf{proxy.conf.json}: Este es el fichero que nos va a ayudar a poder consumir nuestra API. Reenvía las peticiones que llegan a nuestra aplicación al puerto 3000 que es donde se encuentra ubicada.
\end{itemize}
En la misma carpeta raíz tenemos un directorio llamado \textit{src}, su descomposición es la siguiente:

\begin{itemize}
    \item \textbf{app}: Directorio que contiene todos los distintos componentes de los que está compuesta la aplicación.
    \item \textbf{assets}: Contiene imágenes y otros recursos para ser copiados en el momento que se construya la aplicación.
    \item \textbf{environments}: Gracias a este fichero podemos configurar una opción en particular de construcción de la aplicación.
    \item \textbf{favicon.ico}: El ícono que sale en la parte superior de la pestaña de la página web.
    \item \textbf{index.html}: La página principal que tiene cualquier web. El CLI se dedica a añadir automáticamente todo el JavaScript y el CSS cuando construye la aplicación. No es un fichero que se use.
    \item \textbf{main.ts}: Este fichero es el punto de entrada principal de la aplicación. Compila la aplicación y arranca el módulo raíz de la aplicación (AppModule) para que se ejecute en el navegador.
    \item \textbf{polyfills.ts}: Provee de adaptaciones para distintos navegadores.
    \item \textbf{styles.css}: Es un archivo de configuración global de estilos para todos los componentes de la aplicación.
    \item \textbf{test.ts}: El punto de entrada principal para los test que se realicen en la aplicación.
\end{itemize}

El contenido del directorio \textit{app} en el momento de la creación del proyecto es el siguiente:

\begin{itemize}
    \item \textbf{app.component.ts}: Define la lógica para la aplicación raíz.
    \item \textbf{app.component.html}: Define el diseño HTML asociaciado con el elemento raíz.
    \item \textbf{app.component.css}: Define el elemento de diseño para el elemento raíz.
    \item \textbf{app.component.spec.ts}: Define el conjunto de pruebas asociado con el elemento raíz.
    \item \textbf{app.module.ts}: Define el módulo raíz, este fichero le comunica a Angular cómo se tiene que realizar el ensamblaje de la aplicación. Inicialmente está declarado detrno de él el propio módulo raíz, pero a medida que vayamos añadiendo elementos a nuestra aplicación irá incluyendo más módulos.
\end{itemize}

Dentro de la carpeta \textit{src} hay tres directorios más:
\begin{itemize}
    \item \textit{components}
    \item \textit{interfaces}
    \item \textit{services}
\end{itemize}