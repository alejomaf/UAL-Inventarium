\section{Creación y puesta en marcha de la Interfaz de Programación de Aplicaciones (API)}
Las APIs son una de esas pequeñas implementaciones que han cambiado enormemente los desarrollos en la informática, hoy en día si tu aplicación no dispone de una API \cite{api-definition} implica varios aspectos:

\begin{itemize}
    \item La expansión de tu aplicación a otras tecnologías conllevará un esfuerzo mayor que si tuviera una API.
    \item Tu servidor requiere más servicios y por tanto recursos para funcionar.
    \item La lógica de negocio de la aplicación ha sido integrada en conjunto con la web presentando una alto nivel de acoplamiento y dificultando la modificación de la misma.
\end{itemize}

Una API ofrece una capa de abstracción para la aplicación. El famoso modelo de caja negra del que se ha hablado con anterioridad. Esto permite a distintas tecnologías orientadas a desarrollo web, móvil y de programas de ordenador a tener un punto en común entre todas. A que cada una no necesite personalizar sus comunicaciones con la base de datos y pueda usar un intermediario. Así es el funcionamiento de las APIs.
\\Para implementar la API se ha utilizado Node junto a Express. Gracias a Node puede levantarse un servidor web que no necesite de demasiados recursos para su funcionamiento. Esto se haría con las siguientes líneas de código:
\begin{verbatim}
    app.listen(port, () => {
        console.log(`Inventarium API listening at http://api:3000`)
    });
\end{verbatim}
Otra de las partes que aporta Node es su gestor de paquetes, Node Package Manager (NPM), gracias a esto la implementación de funcionalidades más complejas se pueden realizar en un tramo de tiempo muy pequeño.
\\Para inicializar un proyecto de NPM se utiliza el siguiente comando dentro de un directorio:
\begin{verbatim}
    npm init
\end{verbatim}

Express \cite{node-express} es unos de los frameworks principales que presenta Node. Es un marco de desarrollo minimalista para Node que nos permite estructurar aplicaciones, crear enrutamientos y muchos más aspectos relacionados con lo que sería un entorno web.
\\Para instalar Express en un proyecto NPM y guardarlo en la lista de dependencias hay que escribir el siguiente comando:
\begin{verbatim}
    npm install express --save
\end{verbatim}

\subsection{Creación del sistema de directorios}
Un apartado a tratar en este capítulo es la gestión de directorios y los distintos ficheros que hay en su ubicación interactúan entre ellos.
\\Ya habiendo creado el proyecto Node y habiéndole añadido Express esta sería la estructura de directorios que se usará:

\begin{itemize}
    \item \textbf{images}: Dentro de este directorio se almacenan las imágenes que se van subiendo al sitio web. Estas imágenes van ligadas a un ``grupo de objetos''.
    \item \textbf{node\_modules}: La carpeta de node\_modules es donde se gestionan todos los paquetes del NPM.
    \item \textbf{routes}: Dentro de la carpeta routers se creará todo el enrutamiento de la aplicación. En ella se ubica un archivo por cada tabla que se presenta en la base de datos.
    \item \textbf{services}: El objetivo de la carpeta services es el de poder realizar todos los tipos de consultas que requiera la aplicación.
          \begin{itemize}
              \item \textit{.dockerignore}: La funcionalidad de este fichero es el de poder ignorar un directorio o fichero en el momento de la creación de una imagen de un contenedor docker. En este caso el único directorio que se está ignorando es node\_modules. ¿Por qué? Porque en el momento de la generación de una imagen docker suele dar problemas el importar directamente las dependencias que se iban a utilizar a mano. Es mejor que desde el propio sistema de gestor de paquetes, después de leer el package\-lock.json, sea quien instale las dependencias nuevamente.
              \item \textit{.env}: Aquí se ubicarán las variables con las que se trabajará en el entorno. Es una forma fácil y sencilla de poder generalizar secciones del código. El contenido del archivo es el siguiente:
                    \begin{verbatim}
            DB_HOST='localhost',
            DB_USER='user',
            DB_PASSWORD='secretpassword',
            DB_NAME='ualinventarium',
        \end{verbatim}
              \item \textit{config.js}: Dentro del fichero de configuración se añadirán los parámetros que va a coger la variable ``db'' para conectarse a la base de datos.
                    \begin{verbatim}
            const config = {
                db: {
                    host: env.DB_HOST,
                    user: env.DB_USER,
                    password: env.DB_PASSWORD,
                    database: env.DB_NAME'
                },
                listPerPage: env.LIST_PER_PAGE || 10,
            };
        \end{verbatim}
                    \begin{tcolorbox}
                        [colback=green!5!white,colframe=green!75!black,fonttitle=\bfseries,title=Utilización del archivo .env]
                        Podemos ver cómo nuestro archivo que genera la configuración para la base de datos solicita la información dentro de nuestro fichero \.env. El último atributo que intenta solicitar es ``LIST\_PER\_PAGE'' en el caso que no lo pueda obtener, que será lo que va a ocurrir, pondrá como valor predeterminado 10.
                    \end{tcolorbox}
              \item \textit{Dockerfile}: Este archivo funciona para la generación de una imagen de un contenedor docker.
              \item \textit{helper.js}: Un pequeño fichero que brinda un par de funciones a la hora de manejar consultas con la base de datos.
              \item \textit{index.js}: El archivo principal de la API, dentro de él se inicializará todo.
              \item \textit{package\-lock.json}: El fichero principal de NPM que ayuda a gestionar todas las dependencias con los paquetes que se tengan instalados en el proyecto.
              \item \textit{package.json}: El paquete json es el corazón de cualquier proyecto de Node. Registra metadatos importantes necesarios para la publicación de la aplicación, y también define atributos funcionales de un proyecto que NPM usa para instalar dependencias, ejecutar scripts e identificar el punto de entrada al paquete.
          \end{itemize}
\end{itemize}

\subsection{Funcionamiento de index.js}

Dentro de index.js se gestionarán todas las llamadas a los diferentes componentes de la API. Desde añadir el ruteo hasta gestionar la utilización de diferentes librerías que se hayan instalado dentro del proyecto.
\\Una variable importante a recalcar es la de Express. La cual se define así:
\begin{verbatim}
    const app = express();
\end{verbatim}
Gracias a la variable ``app'' se puede enlazar toda la configuración principal del ruteo que va a tener la API.
\\Por ejemplo, de la ruta \textit{/api/grupoobjetos} se quiere que el usuario pueda hacer interacciones con la tabla de grupo de objetos. Para gestionarlo se enlaza ese punto al archivo de ruteo que se explicará en el siguiente apartado:
\begin{verbatim}
    //La variable grupoobjetos es la referenciación del archivo de ruteo
    app.use('/api/grupoobjetos', grupoobjetos); 
\end{verbatim}
Se realizaron otras gestiones en el componente de Express entre las que se encuentran:
\begin{itemize}
    \item Aumentar el tamaño de las peticiones que lleguen para la carga de imágenes.
    \item Configurar un endpoint en la raíz del servidor que tenga como finalidad comprobar que la API funciona correctamente.
\end{itemize}
Por último se configuró el puerto de entrada del servidor:
\begin{verbatim}
    app.listen(port, () => {
        console.log(`Inventarium API listening at http://api:3000')
    });
\end{verbatim}
Esto permite que la aplicación se encuentre ``escuchando'' cada petición de entrada que le envíe la página web.

\subsection{Enrutamiento de la aplicación}
El objetivo del directorio ``routes'' es la correcta gestión de todas las peticiones para el funcionamiento de la página web.
\\Cuando se crea una herramienta para la gestión de los recursos de una base de datos lo más normal es que se necesite implementar el repertorio de herramientas llamadas CRUD, create, read, update and, delete, es decir: crear, leer, actualizar y eliminar.
\\Al inicio del archivo de enrutamiento se llama a la funcionalidad de Router que incopora Express:
\begin{verbatim}
    const router = express.Router();
\end{verbatim}

\subsubsection{Métodos de petición HTTP}
Los métodos de petición HTTP es la definición de un conjunto de elementos que tiene como objetivo realizar diferentes acciones para la gestión de un recurso determinado.
\\Estos métodos serán los que ayuden en la implementación del repertorio de herramientas CRUD.

\subsubsection{GET (leer)}
El método GET es el encargado de solicitar un recurso en específico. Estas peticiones, por norma general, deben tener como único objetivo la recopilación de datos.
\\Por ejemplo, en el archivo de grupoobjetos se implementa el siguiente método:
\begin{verbatim}
/* GET group_of_objects. */
router.get('/', async function (req, res, next) {
    try {
        res.json(await group_of_objects.getMultiple(req, req.query.page));
    } catch (err) {
        console.error(`Error while getting group_of_objects `, err.message);
        next(err);
    }
});
\end{verbatim}
Dentro de la llamada al método ``get'' de ``router'' se realiza un ``try catch'' en el que se llama al método ``getMultiple'' de la clase que se ha importado con anterioridad de group\_of\_objects con el siguiente método:
\begin{verbatim}
    const group_of_objects = require('../services/grupo_objetos');
\end{verbatim}
El método ``getMultiple'' devuelve una lista con los diferentes grupos de objetos que hay en el servidor. Se ahondará más en ese método en la siguiente sección.
\\Dentro del ``catch'' se hace que la API devuelva un mensaje de error como respuesta en caso de fallo.
\begin{tcolorbox}
    [colback=green!5!white,colframe=green!75!black,fonttitle=\bfseries,title=Personalización de peticiones]
    El endpoint que gestiona ``grupoobjetos'' es \textit{/api/grupoobjetos} por lo que al estar definiendo dentro del router.get el parámetro \textit{`/'} se pretende que la API ofrezca esa petición desde la raíz del endpoint. Por ejemplo, si se quisiera personalizar más la petición podría modificarse este apartado y en vez de usar \textit{`/'} se usaría \textit{`/fungibles'}. Para poder acceder a esa consulta GET habría que llamar a \textit{/api/grupoobjetos/fungibles}.
\end{tcolorbox}
No hace falta que se le pase ningún parámetro a la petición GET pero se ha dejado en caso de realizar alguna implementación extra.

\subsubsection{POST (escribir)}
El método POST se utilizará para el envío de una entidad a un determinado recurso. La API hará uso de este método para la adición de elementos a la base de datos.

\subsubsection{PUT (actualizar)}
Este método se encarga del reemplazo de una determinada entidad. Se utilizará para la actualización de las filas de la base de datos.

\subsubsection{DELETE (eliminar)}
Como su propio nombre indica en inglés el método DELETE se utilizará para borrar un recurso en específico.

\vspace{\baselineskip}
Se ha añadido solamente un ejemplo dentro del método GET porque el resto de llamadas se realiza de forma muy parecida. En el POST y PUT se pasarían objetos como parámetros y en el PUT se especificaría una ID para la actualización del objeto. En DELETE solamente se especificaría la ID al igual que dentro del método PUT. Por ejemplo, si se quiere realizar una llamada eliminando el grupo de objetos con id 4 se llamaría al método DELETE de la API con la siguiente dirección \textit{/api/grupoobjetos/4}.

\subsection{Consultas con la base de datos}
Dentro de la carpeta ``services'' están las consultas con la base de datos. En ella se definen todas las posibles interacciones que querría realizar el usuario con la base de datos.
\\Las funciones coincidentes en todos los ficheros que se encuentran ubicados en este directorio son las siguientes:
\begin{itemize}
    \item \textbf{getMultiple}: dentro de ella se llama al método SELECT de MariaDB.
    \item \textbf{create}: se llama al método INSERT INTO.
    \item \textbf{update}: se llama al método UPDATE.
    \item \textbf{remove}: se llama al método DELETE.
\end{itemize}
Para explicar la estructura de estas funciones se cogerá como ejemplo ``getMultiple'':
\begin{verbatim}
    const offset = helper.getOffset(page, config.listPerPage);
    const rows = await db.query(
      `SELECT idGrupoObjetos, cantidad, nombre, imagen, marca, 
      modelo, cantidadDisponible, tipo, eliminado 
      FROM grupoobjetos WHERE eliminado = 0 LIMIT ?,?`,
      [offset, config.listPerPage]
    );
    const data = helper.emptyOrRows(rows);
    const meta = { page };
  
    return {
      data,
      meta
    }
  }
\end{verbatim}
Como puede comprobarse se llama al método offset que ayudará a realizar la paginación del sitio web.
\\Después de la consulta se llama al método de la clase definida anteriormente ``helper'' llamado ``emptyOrRows''. Este método ayuda a evitar cualquier problemática que pueda causar la API si la base de datos devuelve el array vacío.
\\Por último en el ``return'' de la función se devuelve un JSON que tiene como primer elemento la respuesta de la petición y como segundo la página.

\subsection{Gestión de archivos con Express}

Para poder realizar la subida de imágenes a la base de datos se hará uso de la librería ``formidable''.
\\Esta se encontrará importada en el fichero ``index.js'' de la siguiente forma:

\begin{verbatim}
    const formidable = require('express-formidable');
\end{verbatim}

Se procederá luego a incorporarla a Express de la siguiente forma:

\begin{verbatim}
    app.use(formidable());
\end{verbatim}

Gracias a esto ya puede analizarse el campo de archivos de las peticiones que lleguen.

\subsubsection{Descomposición de la petición}
Dentro de la petición que llega a la API se coge el campo de archivos que va vinculada a ella gracias a la utilización de \textit{formidable}.
\begin{verbatim}
    files = req.files
\end{verbatim}

\subsubsection{Análisis del archivo}
Luego se procede a analizar el archivo para que cumpla los siguientes requisitos:
\begin{itemize}
    \item Que sea únicamente un solo archivo.
    \item El método ``isFileValid'' comprueba que el archivo tenga la extensión \textit{jpg}, \textit{jpeg}, \textit{png}. Es decir, que sea una imagen. Se ha seleccionado únicamente esta extensión de archivos porque luego si se cambia la extensión a \textit{jpg} se pueden visualizar sin problemas.
\end{itemize}

\begin{verbatim}
    // Se comprueba que el archivo sea uno o más de uno
    if (!files.length) {
      //En el caso de que sea solo un archivo
      const file = files.image;
      // Se comprueba que el archivo sea válido
      const isValid = isFileValid(file);
      // Se crea un nombre en base al momento actual en que se
      // está subiendo el archivo
      const fileName = time + ".jpg";

      if (!isValid) {
        // Si el archivo no es válido se lanza un error
        return res.status(400).json({
          status: "Fail",
          message: "The file type is not a valid type",
        });
      }
\end{verbatim}

\subsubsection{Subida de la imagen}
Gracias a esto puede realizarse la subida de la imagen sin problemas. El directorio de subida se será ``images''.

\begin{verbatim}
      const uploadFolder = path.join(__dirname, "..", "images",
       "group_of_objects");
      try {
        // Se cambia el nombre del archivo en el directorio
        fs.renameSync(file.path, path.join(uploadFolder, fileName));
      } catch (error) {
        console.log(error);
      }

    } else return;
\end{verbatim}

\subsubsection{Envio de la consulta a la base de datos}
Se envía una solicitud a la base de datos para finalizar con la creación del grupo de objeto. En caso que la subida diera error este proceso se pararía. Como puede comprobarse, al llamar al método de creación de grupo de objetos se le pasa como parámetro el campo \textit{fields}, donde va el cuerpo de la petición.

\begin{verbatim}
    //Consulta post en la base de datos
    res.json(await group_of_objects.create(req.fields, time));


  } catch (err) {
    console.error(`Error while creating group of objects`, err.message);
    next(err);
  }
});
\end{verbatim}

Ya con esto se tendría la imagen subida con el nombre del archivo actualizado dentro de la base de datos.
