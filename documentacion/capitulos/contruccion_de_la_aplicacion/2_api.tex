\section{Creación y puesta en marcha de la Interfaz de Programación de Aplicaciones (API)}
Las APIs son una de esas pequeñas implementaciones que han cambiado enormemente los desarrollos en la informática, hoy en día si tu aplicación no dispone de una API implica varios aspectos:

\begin{itemize}
    \item Tu aplicación no presenta una separación cliente servidor. Esto quiere decir que su modularidad es baja, en caso de querer realizar cambios sobre la aplicación estos van a ser complicados de acometer. En otro caso de que tu aplicación se quiera expandir a una nueva tecnología esta tendrá que volver a implementar otro modelo de comunicación con la base de datos.
    \item La expansión de tu aplicación a otras tecnologías conllevará un esfuerzo mayor que si tuviera una API.
    \item Tu servidor requiere más servicios y por tanto recursos para funcionar.
    \item La lógica de negocio de la aplicación ha sido integrada en conjunto con la web presentando una alto nivel de acoplamiento y dificultando la modificación de la misma.
\end{itemize}

Una API nos ofrece una capa de abstracción para nuestra aplicación. Nuestro famoso modelo de caja negra del que he hablado con anterioridad. Esto permite a distintas tecnologías orientadas a desarrollo web, móvil y de programas de ordenador a tener un punto en común entre todas. A que cada una no necesite personalizar sus comunicaciones con la base de datos y pueda usar un intermediario. Así es el funcionamiento de las APIs.
\\Para implementar nuestra API hemos utilizado Node junto a Express. Gracias a Node podemos levantar un servidor web que no necesite de demasiados recursos para su funcionamiento. Esto se haría con las siguientes líneas de código:
\begin{verbatim}
    app.listen(port, () => {
        console.log(`Inventarium API listening at http://api:3000`)
    });
\end{verbatim}
Otra de las partes que nos aporta Node es su gestor de paquetes, Node Package Manager (NPM), gracias a esto la implementación de funcionalidades más complejas se pueden realizar en un tramo de de tiempo muy pequeño.
\\Para inicializar un proyecto de NPM utilizamos el siguiente comando dentro de un directorio:
\begin{verbatim}
    npm init
\end{verbatim}

Express es unos de los frameworks principales que presenta Node. Es un marco de desarrollo minimalista para Node que nos permite estructurar aplicaciones, crear enrutamientos y un muchos más aspectos relacionados con lo que sería un entorno web.
\\Para instalar Express en un proyecto NPM y guardarlo en la lista de dependencias escribimos el siguiente comando:
\begin{verbatim}
    npm install express --save
\end{verbatim}

\subsection{Creación del sistema de directorios}
Un apartado a tratar en este capítulo es la gestión de directorios y como los distintos ficheros que hay en su ubicación interactúan entre ellos.
\\Ya habiendo creado nuestro proyecto Node y habiéndole añadido Express esta sería la estructura de directorios que usaríamos:

\begin{itemize}
    \item \textbf{images}: Dentro de este directorio almacenamos las imágenes que se van subiendo a nuestro sitio web. Estas imágenes van ligadas al tipo de dato ``grupo de objeto''.
    \item \textbf{node\_modules}: La carpeta de node\_modules es donde se gestionan todos los paquetes de nuestro NPM.
    \item \textbf{routes}: Dentro de la carpeta routers crearemos todo el enrutamiento de nuestra aplicación. En ella se ubica un archivo por cada tabla que se nos presenta en la base de datos.
    \item \textbf{services}: El objetivo de la carpeta services es el de poder realizar todos los tipos de consultas que requiera la aplicación.
    \begin{itemize}
        \item \textit{\.dockerignore}: La funcionalidad de este fichero es el de poder ignorar un directorio o fichero en el momento de la creación de una imagen de un contenedor docker. En este caso el único directorio que estamos ignorando es node\_modules. ¿Por qué? Porque en el momento de la generación de una imagen docker suele dar problemas el importar directamente las dependencias que se iban a utilizar a mano. Es mejor que desde el propio sistema sea el gestor de paquetes quien, después de leer el package\-lock.json, sea quien instale las dependencias nuevamente.
        \item \textit{\.env}: Aquí ubicamos las variables con las que trabajaremos en nuestro entorno. Es una forma fácil y sencilla de poder generalizar secciones del código. El contenido del archivo es el siguiente:
        \begin{verbatim}
            DB_HOST='localhost',
            DB_USER='user',
            DB_PASSWORD='secretpassword',
            DB_NAME='ualinventarium',
        \end{verbatim}
        \item \textit{config.js}: Dentro del fichero de configuración añadimos los parámetros que va a coger nuestra variable db para conectarse a la base de datos.
        \begin{verbatim}
            const config = {
                db: {
                    host: env.DB_HOST,
                    user: env.DB_USER,
                    password: env.DB_PASSWORD,
                    database: env.DB_NAME'
                },
                listPerPage: env.LIST_PER_PAGE || 10,
            };
        \end{verbatim}
        \begin{tcolorbox}
            [colback=green!5!white,colframe=green!75!black,fonttitle=\bfseries,title=Utilización del archivo .env]
            Podemos ver como nuestro archivo que genera la configuración para la base de datos solicita la información dentro de nuestro fichero \.env. El último atributo que intenta solicitar es ``LIST\_PER\_PAGE'' en el caso que no lo pueda obtener, que será lo que va a ocurrir, pondrá como valor predeterminado 10.
        \end{tcolorbox}
        \item \textit{Dockerfile}: Este archivo funciona para la generación de una imagen de un contenedor docker.
        \item \textit{helper.js}: Un pequeño fichero que nos brinda un par de funciones a la hora de manejar consultas con la base de datos.
        \item \textit{index.js}: El archivo principal de nuestra API, dentro de él inicializaremos todo.
        \item \textit{package\-lock.json}: El fichero principal de NPM que nos ayuda a gestionar todas las dependencias con los paquetes que tenemos instalados en nuestro proyecto.
        \item \textit{package.json}: El paquete json es el corazón de cualquier proyecto de Node. Registra metadatos importantes sobre un proyecto que necesarios para la publicación de la aplicación, y también define atributos funcionales de un proyecto que npm usa para instalar dependencias, ejecutar scripts e identificar el punto de entrada a nuestro paquete.
    \end{itemize}
\end{itemize}

\subsection{Funcionamiento de index.js}
Dentro

\subsection{Enrutamiento de la aplicación}

\subsection{Consultas con la base de datos}

\subsection{Configuracion del archivo Dockerfile}